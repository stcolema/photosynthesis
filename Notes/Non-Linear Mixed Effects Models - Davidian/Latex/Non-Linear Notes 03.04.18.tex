% !TEX TS-program = pdflatex
% !TEX encoding = UTF-8 Unicode

% This is a simple template for a LaTeX document using the "article" class.
% See "book", "report", "letter" for other types of document.

\documentclass[11pt]{article} % use larger type; default would be 10pt

\usepackage[utf8]{inputenc} % set input encoding (not needed with XeLaTeX)

%%% Examples of Article customizations
% These packages are optional, depending whether you want the features they provide.
% See the LaTeX Companion or other references for full information.

%%% PAGE DIMENSIONS
\usepackage{geometry} % to change the page dimensions
\geometry{a4paper} % or letterpaper (US) or a5paper or....
% \geometry{margin=2in} % for example, change the margins to 2 inches all round
% \geometry{landscape} % set up the page for landscape
%   read geometry.pdf for detailed page layout information

\usepackage{graphicx} % support the \includegraphics command and options

% \usepackage[parfill]{parskip} % Activate to begin paragraphs with an empty line rather than an indent

%%% PACKAGES
\usepackage{booktabs} % for much better looking tables
\usepackage{array} % for better arrays (eg matrices) in maths
\usepackage{paralist} % very flexible & customisable lists (eg. enumerate/itemize, etc.)
\usepackage{verbatim} % adds environment for commenting out blocks of text & for better verbatim
\usepackage{subfig} % make it possible to include more than one captioned figure/table in a single float
\usepackage{amsmath} % AMS package for various mathematical notation
\usepackage{amsfonts} % contains mathbb and other nice fonts

% These packages are all incorporated in the memoir class to one degree or another...

%%% HEADERS & FOOTERS
\usepackage{fancyhdr} % This should be set AFTER setting up the page geometry
\pagestyle{fancy} % options: empty , plain , fancy
\renewcommand{\headrulewidth}{0pt} % customise the layout...
\lhead{}\chead{}\rhead{}
\lfoot{}\cfoot{\thepage}\rfoot{}

%%% SECTION TITLE APPEARANCE
\usepackage{sectsty}
\allsectionsfont{\sffamily\mdseries\upshape} % (See the fntguide.pdf for font help)
% (This matches ConTeXt defaults)

%%% ToC (table of contents) APPEARANCE
%\usepackage[nottoc,notlof,notlot]{tocbibind} % Put the bibliography in the ToC
\usepackage[titles,subfigure]{tocloft} % Alter the style of the Table of Contents
\renewcommand{\cftsecfont}{\rmfamily\mdseries\upshape}
\renewcommand{\cftsecpagefont}{\rmfamily\mdseries\upshape} % No bold!

%%% END Article customizations

%%% The "real" document content comes below...


% BibTex packages (url for website references)
\usepackage[english]{babel}
\usepackage[numbers]{natbib}
\usepackage{url}

%For inclusion of hyperlinks
\usepackage{hyperref}
\hypersetup{
    colorlinks=true,
    linkcolor=blue,
    filecolor=magenta,      
    urlcolor=cyan,
}

%BibTex stuff and referencing sections by name 
\urlstyle{same}
\usepackage{nameref} 

%Suggested solution to large spacing in bulletpoints
%\usepackage{flafter}

% define a "Definition" environment
\newenvironment{definition}[1][Definition]{\begin{trivlist}
\item[\hskip \labelsep {\bfseries #1}]}{\end{trivlist}}

% define more concise new commands
\newcommand{\bbeta}{\boldmath{$\beta$ }\unboldmath }
\newcommand{\bepsilon}{\boldmath{$\epsilon$ }\unboldmath }
\newcommand{\bmu}{\boldmath{$\mu$ }\unboldmath }
\newcommand{\bmui}{\boldmath{$\mu$}\unboldmath $_i$ }

\newcommand{\bpi}{\boldmath{$\pi$ }\unboldmath }

% command for oversetting distributed sign with text
\makeatletter % changes the catcode of @ to 11 (i.e. letter)

\newcommand{\distas}[1]{\mathbin{\overset{#1}{\kern\z@\sim}}}%
\newsavebox{\mybox}\newsavebox{\mysim}

\newcommand{\distras}[1]{%
  \savebox{\mybox}{\hbox{\kern3pt$\scriptstyle#1$\kern3pt}}%
  \savebox{\mysim}{\hbox{$\sim$}}%
  \mathbin{\overset{#1}{\kern\z@\resizebox{\wd\mybox}{\ht\mysim}{$\sim$}}}%
}
\makeatother % changes the catcode of @ to 12 (i.e. back to default, other)

\title{Non-Linear Mixed Effects Models \\ Davidian \& Giltinan}
\author{Stephen Coleman}
%\date{} % Activate to display a given date or no date (if empty),
         % otherwise the current date is printed 

\begin{document}
\maketitle

%\newpage
%\tableofcontents
%\newpage

\section{Introduction}
\subsection{What, when and why}
Non-linear mixed effects models, also known as \emph{non-linear hierarchical models}, are an extension to the more traditional linear mixed-effects models. They are used in scenarios where all of the following features are present \cite{DavidianNonlinearmodelsrepeated2003}:

\begin{enumerate}  \label{when_nlme}
 \item Repeated observations of a continuous variable on each of several \emph{experimental units} (in our case these are individuals, thus individual is considered equivalent to empircial unit in the following section) over time or another condition (e.g. measurements at given heights on a tree) (the \emph{condition variable});
 \item We expect the relationship between the response variable and the condition variable to vary across individuals; and
 \item Availability of a scientifically relevant model characterising the behaviour of the individual response in terms of meaningful parameters that vary across individuals and dictate variation in patterns of condition-response (for us this will be the Farquhar-van Cammerer-Berry model).
\end{enumerate}
The final point from the list in \ref{when_nlme} is where the non-linear aspect is introduced. This is often a mechanistic function describing a physical or chemical system (for example in toxicokineitcs physiologically-based pharmacokinetics models are used or HIV dynamics in ``precision medicine''); that is the model is described by by meaningful, interpretable parameters rather than being an empirical best fit.

The analysis tends to have the goal of understanding one or more of the following:

\begin{enumerate}
 \item The ``typical'' behaviour of the phenomena (i.e. mean or median values) represented by the model parameters;
 \item The variation of these parameters, and hence the phenomena, between individuals; and
 \item If some of the variation is inherently associated with individual characteristics.
\end{enumerate}
Individual level prediction can also be of interest (e.g. in medical treatment with highly individual reaction), but is less relevant to this project. From the individual level we are interested in investigating the level of variation between individuals and questioning if this is sufficiently small to allow the ``all-purpose'' models generally used in photosynthesis describing the parameters and systems of interest.

\subsection{The Model}
\subsubsection{Basic model}
Consider an experiment involving repeated measurements of some response variable, $Y$, across a condition variable $T$ for $n$ individuals. Each individual's characteristics are recorded in $A$ (assumed to be scleronemous) and possible additional initial conditions in $U$. Let $y_{i,j}$ denote the $j$th measurement of the response under condition $t_{i,j}$, $j = 1,\ldots,n_i$, for individual $i$, $i = 1,\ldots,n$. Thus:

\begin{equation}
\begin{array}{ll}
Y &= \bigcup\limits_{i=1}^n y_i\\
y_i &= \left\{y_{i,1},\ldots,y_{i,n_i}\right\} \\
T &= \bigcup\limits_{i=1}^n t_i \\
t_i &= \left\{t_{i,1},\ldots,t_{i,n_i}\right\} \\
A &= \left\{a_i,\ldots,a_n\right\} \\
U &= \left\{u_1,\ldots,u_n\right\}
\end{array}
\end{equation}
Often $T$ is time and $U=\emptyset$, but it might be the case that $T$ is an increase in some environmental condition and $U$ might be some meta-data available for each of the individuals (for example genotype data in the case of ``precision medicine''). For brevity's sake use $x_{i,j} = \left(t_{i,j}, u_i\right)$. The assumption that $(y_i,u_i,a_i)$ are independent across $i$ is often included to reflect the belief that individuals are unrelated (this will hold for us, but might require more thought in other situations). For some function $f$ regulating the within-individual behaviour defined by a vector of parameters $\beta_i$ unique to individual $i$, we have:
\begin{equation} \label{stage1_individual_model}
y_{i,j} = f\left(x_{i,j}, \beta_i\right) + \epsilon_{i,j}, \hspace{4mm} j = 1,\ldots,n_i
\end{equation}
We assume that $\mathbb{E}\left(e_{i,j}|u_i,\beta_i\right)=0$ for all $i, j$. (\ref{stage1_individual_model}) is called the \emph{individual level model}. To model the population parameters we consider $d$, a $p$-dimensional function depending on an $r$-vector of fixed parameters, or \emph{fixed effects}, $\beta$, and a $k$-vector of \emph{random effects}, $b_i$, associated with individual $i$:

\begin{equation} \label{stage1_population_model}
\beta_i = d\left(a_i,\beta,b_i\right), \hspace{4mm} i = 1,\ldots,n
\end{equation}
Here, the \emph{population model} in (\ref{stage1_population_model}) describes how $beta_i$ varies among individuals due to both individual attributes $a_i$ and biological variation in $b_i$. We assume that the $b_i$ are independent of the $a_i$, i.e.:

\begin{equation}
\begin{array}{l}
\mathbb{E}(b_i|a_i) = \mathbb{E}(b_i) = 0 \\
\mathbb{V}ar(b_i|a_i) = \mathbb{V}ar(b_i) = D
\end{array}
\end{equation}
Here, $D$ is an unstructured covariance matrix and is common to all $i$. It characterises the degree of unexplained variation in the elements of $\beta_i$ and associations among them; the ubiquitous assumption is $b_i \sim \mathcal{N}(0,D)$. However, if this set of assumptions regarding the conditional distribution of $b_i$ on $a_i$ is found to be insufficient, then $b_i \sim \mathcal{N}\left(0,D(a_i)\right)$ is frequently used.

In (\ref{stage1_population_model}), $\beta_i$ is considered to have an associated random effect, reflecting the belief that each component varies non-negligibly in the population even after systematic relationships with subject characteristics are accounted for. It may happen that ``unexplained'' variance in a component of $\beta_i$ may be very small in magnitude relative to that in the remaining elements. In this situation it is common to drop the negligible quantity entirely. This lacks biological sense as each parameter is part of the ``scientifically relevant model'' and thus is unlikely to have no associated unexplained variation. Hence, one must recognise that this omission of an element of $\beta_i$ is adopted to achieve numerical stability in fitting rather than to reflect belief in perfect biological consistency across individuals and analyses in the literature to determine whether elements of $\beta_i$ are fixed or random effects should be interpreted so.

%\newpage
%\bibliographystyle{unsrt}
\bibliographystyle{plainnat}
%\bibliographystyle{te}
\bibliography{References}



\end{document}